\subsection{Global Trajectory and Global Map Generation}
The global trajectory is obtained directly from the SLAM back end. The optimizer outputs each estimated vehicle pose in the Lie group SE(3), which already encodes the full 3D rotation and translation of the platform. Generating the trajectory therefore becomes trivial, see Equation \eqref{eq:system-modeling-se3-transform-structure}. Each optimized pose is appended to a trajectory buffer and plotted in the global frame without additional computation. No filtering or smoothing is required at this stage, since all corrections (including loop closures and re-linearization updates) are already reflected in the optimizers output.
\\ \\
The global map generation builds on the same principle. Each local Cartesian map tile originates from the local map pipeline, where slant range correction, pitch and roll compensation, probabilistic interpolation, and normalization have already been applied. Because these tiles are internally consistent and represented in a local body centered frame, the global layer only needs to apply the corresponding SE(3) pose to transform every pixel into the global coordinate system, see Equation \eqref{eq:system-modeling-se3-transform-point}. Conceptually, this is identical to standard computer graphics rendering, take an image patch, apply a rigid body transform, and place it in a world canvas.
\\ \\
As the vessel moves, thousands of local maps accumulate. To manage this efficiently, a KD tree or similar kNN based spatial structure can be used to index pixel locations and enable fast insertion, lookup, and updates as the map grows. This becomes especially important when evaluating overlapping regions between newer and older map tiles. Due to the sliding window behaviour described in the \textit{``Local Map Generation''} chapter, where each new tile shares approximately 1/3rd overlap with the previous tile, the global map frequently receives multiple pixel estimates for the same global location. Since the local maps are already intensity normalized and internally smoothed, the global fusion strategy can remain simple, whenever two or more pixels fall into the same global cell, their values are averaged. This produces smooth transitions between overlapping tiles without introducing additional bias or artifacts.
\\ \\
Because all heavy preprocessing has already been performed in local map generation, the global map does not require probabilistic fusion, filtering, or advanced occupancy grid models. Its role is strictly compositional, transform, place, and blend. The final structure can also be stored in a lightweight database or grid based memory layout to support efficient post processing or rapid access during mission playback. This allows the global map to act not only as a visual tool but also as a reusable data product ready for integration into autonomy frameworks such as obstacle avoidance, seafloor classification, semantic mapping, and long term planning tasks.
\\ \\
Overall, the global trajectory and map module is intentionally simple. The optimizer provides accurate SE(3) poses, the local map generation pipeline provides consistent Cartesian tiles, and the global layer merely stitches them together into a coherent and continuously growing world representation. This results in a robust and computationally efficient final stage of the SSS SLAM system.



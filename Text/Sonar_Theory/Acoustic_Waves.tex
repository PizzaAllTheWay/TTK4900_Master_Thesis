\subsection{Acoustic Waves}
Sound in water travels as longitudinal pressure waves, generated by small compressions and rarefactions of the surrounding medium. Unlike electromagnetic waves, acoustic waves propagate efficiently through water due to its density and elasticity, resulting in long range transmission and relatively low attenuation. The speed of sound in seawater is approximately $1500~\mathrm{m/s}$, depending on temperature, salinity, and depth. These parameters slightly modify the propagation path and timing, but for most practical applications in coastal robotics, the speed is assumed constant.
\\ \\
Acoustic waves follow simple geometric spreading, as they travel outward, energy is distributed over a larger area, causing the received amplitude to decrease with range. Additional attenuation arises from absorption in the water and scattering from suspended particles or biological matter. When a sound pulse reaches the seafloor or an object, part of the energy reflects back toward the sensor. The strength and timing of this reflection depend on material properties, surface roughness, angle of incidence, and the distance traveled. These principles form the basis of sonar imaging, where the measured return intensity encodes information about the environment.
\\ \\
Because sonar relies on timing and amplitude of returning echoes, understanding propagation and reflection is essential for interpreting range measurements, correcting distortions, and aligning sonar data with vehicle motion. The raw acoustic behavior directly influences the quality and structure of the maps produced later.
\begin{figure}[H]
    \centering
    \includegraphics[width=1.0\linewidth]{Pictures/Sonar_Theory/Acoustic_Waves/Acoustic_Wave_Propagation.png}
    \caption{Illustration of acoustic wave propagation, showing how sound travels through a medium as alternating regions of compression and rarefaction. These pressure variations move outward as a longitudinal wave and can be detected by underwater piezoelectric ceramic transducers.\textsuperscript{\cite{acoustic_waves}}}
    \label{fig:sonar-theory-sonar-acoustic-waves}
\end{figure}


\subsection{Probabilistic Methods}
Probabilistic data association methods attempt to handle ambiguity explicitly by reasoning over multiple possible landmark matches rather than committing to a single nearest neighbour. They represent association uncertainty with probability distributions and update the SLAM system using weighted or multi hypothesis formulations. While powerful in theory, these methods become computationally heavy and often numerically fragile when applied online in larger SLAM problems with many landmarks. \cite{sensor_fusion_book}
\\ \\
PDA (Probabilistic Data Association) distributes probability mass across all gated candidates instead of selecting a single landmark, updating the filter with a weighted combination of them. This soft assignment can handle mild ambiguity but becomes unreliable once many landmarks fall inside the gate, and its computational cost scales poorly without offering clear benefits for sparse seabed features.
\\ \\
JIPDA (Joint Integrated Probabilistic Data Association) combines the ideas from JPDA and IPDA, extending PDA by evaluating all measurement to landmark assignment combinations jointly rather than treating matches independently. This reduces the greediness of standard PDA, but it also requires a highly accurate probabilistic model of landmark geometry and measurement behaviour. For sonar based features, landmark shapes are often ambiguous, weakly defined, and far from Gaussian, making these likelihood models hard to specify correctly and easy to mis tune making DA to overconfident. In practice this leads to inconsistent or low quality associations, while the added mathematical and computational complexity offers little benefit for the SSS SLAM setting. These methods can work in controlled environments, but for seabed mapping they are simply not a suitable fit.
\\ \\
MHT (Multi Hypothesis Tracking) keeps several data association hypotheses alive in parallel and prunes weaker ones over time. It is considered the gold standard for ambiguous tracking problems, but its hypothesis tree grows exponentially with ambiguity, making it usable only offline or with extreme pruning. For SLAM it is typically far too slow for real time use.
\\ \\
Particle filter approaches such as FastSLAM embed data association within each particle so different hypotheses can coexist naturally. While attractive conceptually, SLAM particle filters are known to degrade due to sample depletion, degeneracy, and reliance on accurate proposal distributions. These weaknesses make them fragile for sonar SLAM where measurement noise is high and data arrival is dense.
\\ \\
Overall, although probabilistic methods offer richer representations of association uncertainty, they do not scale well and perform poorly in real time for the sparse but high throughput setting of SSS SLAM. For these reasons they are not adopted in this thesis, where gating combined with NN DA algorithm provides a simpler and more computationally efficient solution.


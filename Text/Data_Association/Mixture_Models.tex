\subsection{Mixture Models}
Mixture model approaches handle data association uncertainty by allowing several competing hypotheses to coexist rather than forcing a single discrete match. Instead of selecting one landmark, the front-end of SLAM creates a mixture factor that encodes multiple possible associations, and each hypothesis is carried into the back-end of SLAM as part of a unified factor. The SLAM optimizer then resolves the ambiguity by selecting the hypothesis that forms the most globally consistent solution with the rest of the graph. This avoids premature front-end decisions and provides a theoretically elegant way to handle ambiguous or noisy measurements. \cite{mixture_models}
\\ \\
A variety of methods exist within this family, including Max Mixture models and MMSAM (Mixture Model Smoothing and Mapping). These frameworks extend classical smoothing by embedding multiple measurement models inside each factor, allowing the optimizer to choose between them. Conceptually this is appealing, no association is thrown away early, ambiguity is preserved, and the final solution is determined by full nonlinear optimization rather than by a brittle nearest neighbour rule. For inherently uncertain sensing domains, such as sonar or radar, mixture models represent robust formulations of DA.
\\ \\
However, according to Haralstad master thesis \cite{side_scan_sonar_master_thesis}, mixture models introduce a severe practical limitation, mixture factors destroy the sparsity structure required for efficient incremental solvers that modern optimizers in SLAM back-end use. Because each mixture factor couples several hypotheses, the factor graph becomes substantially denser, and incremental smoothing (iSAM type methods) can no longer operate efficiently. MMSAM and related mixture smoothing techniques therefore fall back to full batch smoothing methods (SAM), which must repeatedly rebuild and solve large linear systems. Even though each sonar 2D local map frame contains only a handful of landmarks, the repeated construction and elimination of dense mixture factors cause computational costs to grow rapidly, making online operation infeasible.
\\ \\
For these reasons, mixture model DA is not used in this thesis. While conceptually powerful and well suited for offline or post processing scenarios, mixture smoothing does not meet the real-time constraints of SSS SLAM.



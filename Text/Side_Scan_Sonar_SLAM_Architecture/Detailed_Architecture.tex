\subsection{Detailed Architecture}
\subsubsection{Side Scan Sonar Pipeline}
\begin{figure}[H]
    \centering
    \includegraphics[width=0.65\linewidth]{Pictures/Side_Scan_Sonar_SLAM_Architecture/Detailed_Architecture/Full.png}
    \caption{Picture shows the full Side Scan Sonar Simultaneous Localization And Mapping (SSS SLAM) pipeline from raw sensor data to the final global map and estimated trajectory. This diagram expands on the simplified overview by illustrating how hardware, preprocessing, state estimation, local mapping, feature extraction, data association, and the graph based SLAM back end are connected into one continuous system. Each block represents a major subsystem of the SLAM pipeline, and the arrows indicate how information flows from the sensors through processing layers toward the optimizer and the final global reconstruction. This figure acts as a visual ``roadmap'' for the rest of the thesis, helping the reader keep track of where they are as each subsection is later described in more detail.}
    \label{fig:SSS-SLAM-Architecture-Full}
\end{figure}



\clearpage



\subsubsection{Hardware}
The hardware layer consists of three key sensors, the IMU (Inertial Measurement Unit), the GNSS (Global Navigation Satellite System), and the side scan sonar. The GNSS and sonar are exteroceptive sensors, meaning they provide information about the external environment. GNSS is used whenever available to provide absolute positioning and heading corrections, but it cannot be fully relied upon due to multipath, occlusion, or complete signal loss in coastal or obstructed areas. Side scan sonar, on the other hand, forms the primary sensing modality for environmental perception and is therefore central to the SLAM system. The IMU provides continuous high rate inertial measurements, allowing dead reckoning between external sensor updates and ensuring smooth, uninterrupted motion estimation even when GNSS is degraded.
\\ \\
Data from these sensors is typically streamed over Ethernet using vendor specific communication protocols. In the actual system deployment, all raw data is distributed throughout the processing pipeline using ROS2 middleware, ensuring synchronized timestamps, modular development, and reliable communication between nodes running on separate embedded computers. This hardware foundation enables the later stages of the pipeline to fuse complementary information sources into a consistent state estimate and surrounding environment understanding.



\subsubsection{Data Processing}
The data processing stage includes the state estimator and the full local map generation pipeline. The local mapping framework follows the proven structure from prior work by Haraldstad \cite{side_scan_sonar_master_thesis}, which demonstrated reliable performance on real AUV data. The focus of this thesis is to adapt and improve those methods for real-time use. The swath processing stage projects sonar returns into the ground plane, applies pitch and roll corrections, removes blind zones, performs slant range correction, and normalizes intensities. Afterward, probabilistic interpolation and grid reconstruction convert the processed swaths into coherent local Cartesian maps with missing areas filled using interpolation reconstruction techniques. Normalization and smoothing ensure that each local map is consistent and visually stable before landmark extraction.
\\ \\
Feature extraction in this thesis improves upon the earlier work by adopting more robust computer vision techniques. Edge preserving filtering enhances distinct structures while suppressing noise, adaptive thresholding identifies candidate features under varying intensity conditions, and morphological filtering removes weak or noisy artifacts. A connected components (labelling) procedure groups detected features into coherent landmarks. Each landmark is then assigned descriptors that capture its local appearance and spatial context. These descriptors are used later during data association to match landmarks across time.
\\ \\
To generate each local map, the system must know the vehicles pose at each sonar swath. State estimation is therefore tightly integrated with map generation. Among many possible filtering methods, the UKF-M (Unscented Kalman Filter on Manifolds) is selected due to its ease of implementation and its ability to operate without maintaining Jacobians. While an ESKF (Error State Kalman Filter) could also be used, UKF-M provides comparable performance with simpler maintenance and faster convergence in theory. Although UKF-M requires propagating sigma points, modern embedded hardware easily handles this computational load. The result is a stable, real-time, high rate state estimate that supports accurate local map construction.



\subsubsection{SLAM}
The SLAM subsystem is divided into a front-end and a back-end. The front-end is responsible for preparing all measurements that will eventually be inserted into the global factor graph, a global data structure of sort, while the back-end optimizes this global data structure called graph to produce the best possible estimate of both the trajectory and the landmark map. At a high level, the SLAM system takes all incoming information from the sensors, compresses and structures it into meaningful constraints, and feeds these constraints into a graph based optimiser that continuously refines the state estimate over time.
\\ \\
The first major component of the front-end is IMU preintegration. The IMU runs at a very high frequency, often hundreds of measurements per second, which makes it impractical to insert every acceleration and gyro sample directly into the graph. Doing so would create thousands of redundant nodes per minute, overwhelming the optimizer and providing no practical benefit. Instead of relying on a traditional Kalman Filter inside the SLAM loop and propagating a state at every IMU timestamp, preintegration accumulates all IMU data between two keyframes (eks between 2 local sonar 2D maps or 2 GNSS updates) into a single compact factor. This factor encodes the relative motion and uncertainty between the two poses, capturing all the information from hundreds of IMU readings without burdening the optimizer. This design is ideal for real-time SLAM because it preserves the information content of the IMU while keeping the graph compact and manageable.
\\ \\
Direct factors, such as GNSS measurements, are added to the graph whenever available. These factors provide absolute position constraints, which help bound long term drift. When a GNSS update arrives, the preintegration is temporarily paused, the GNSS factor is inserted, and the graph is optimized. Preintegration then resumes from the updated pose. This ensures that the system benefits from GNSS whenever it is reliable, while still functioning robustly in GNSS denied or degraded environments. Sonar landmarks offer a similar anchoring effect. When available, they provide strong positional constraints based on environmental structure, meaning the system does not become dependent on GNSS and can operate independently using only the sonar and IMU.
\\ \\
Data association forms the most challenging and delicate part of the front-end. The goal is to determine whether a newly detected landmark corresponds to one already in the map or whether it represents a new feature. Incorrect data association is extremely harmful, a false positive match can warp the map, distort the trajectory, and corrupt long term consistency. Conversely, overly conservative association can cause the system to miss loop closures, leading to drift. To balance reliability and real-time performance, the system splits data association into two layers, local association and global association.
\\ \\
Before performing any hypothesis testing, the system runs a fast descriptor based matching step using FLANN. This step is not required to be perfect, its purpose is simply to filter a large database of landmarks down to a manageable candidate set. If the descriptors suggest that the new landmarks resemble those observed recently, the system performs local data association. Local data tends to be geometrically consistent, and drift over short time periods is small. For this reason, the system uses JCBB (Joint Compatibility Branch and Bound) with tight gating for local association. JCBB is selected because it provides a strong compromise between accuracy and computational cost. Simpler methods, such as NN (Nearest Meighbour) matching, are fast but too unreliable and prone to incorrect matches. More sophisticated methods, such as MHT (Multiple Hypothesis Tracking) or algorithms like MMSAM (Multi Modal Smoothing and Mapping), are extremely robust but far too slow for real-time operation. JCBB occupies the middle ground, it systematically evaluates compatibility between sets of observations and map landmarks while pruning impossible hypotheses early. This allows it to reject false positives while still maintaining real-time performance on small sets of landmarks.
\\ \\
When the descriptors indicate that a landmark may correspond to something observed long ago, possibly implying a loop closure, the system switches to global JCBB. Global association requires extracting landmark to pose and landmark to landmark covariance information from the back-end, which is significantly slower, but since loop closures are rare compared to local updates, the computational cost remains acceptable. Additionally, a periodic loop closure trigger ensures that the system occasionally attempts a global association even if the FLANN matcher has not suggested a promising candidate. This acts as a safety mechanism to avoid missing opportunities to correct accumulated drift.
\\ \\
The SLAM back-end is responsible for optimizing all constraints collected by the front-end. Many different approaches exist for this, such as EKF-SLAM, FastSLAM, batch Gauss-Newton methods, or sliding-window filters. However, the most successful family of algorithms for modern SLAM systems is smoothing based methods. The key idea behind smoothing is that the optimizer uses all available measurements simultaneously, rather than only maintaining a filtered estimate. This produces far more stable and globally consistent results, especially over long trajectories where drift must be corrected reliably. The challenge, however, is that smoothing requires storing and accessing a very large amount of historical information, which demands an extremely efficient data structure to remain computationally feasible.
\\ \\
In SLAM, this information is most naturally and efficiently represented using a factor graph. A factor graph expresses the probabilistic structure of the problem, variable nodes represent robot poses and landmarks, while factors encode sensor measurements and constraints. Importantly, SLAM factor graphs are highly sparse. Each measurement only depends on a small subset of variables, typically a single pose or a pose to landmark pair. This sparsity reflects the underlying geometry of the problem and is the critical property that makes large scale SLAM solvable in real-time.
\\ \\
To exploit this sparsity, modern smoothing algorithms rely on optimized graph structures that allow fast updates when new data arrives. Among the many available algorithms, iSAM2 has emerged as the most prominent method, especially for landmark based SLAM. iSAM2 exploits the natural sparsity of SLAM factor graphs by reorganizing the graph into an efficient tree structured representation called a Bayes tree. This structure can be solved and updated easily, allowing the optimizer to modify only the affected parts of the solution instead of re-solving the entire graph. As a result, the system remains fast and scalable even as the map grows. iSAM2 is well tested, handles loop closures reliably, and has a mature implementation in GTSAM, making it a practical and robust choice for real-time marine SLAM.
\\ \\
Together, the front-end and back-end form a complete SLAM architecture tailored for real-time marine operations. The front-end carefully selects and structures information so that only high-quality, compact constraints are sent to the optimizer, while the back-end performs continuous refinement to produce a globally consistent trajectory and map. This design ensures robustness, scalability, and real-time capability, which are central goals of this thesis.



\subsubsection{Global Map and Trajectory}
The final stage of the pipeline constructs the global map and estimated trajectory. Each optimized pose is used to transform the corresponding local map into a shared global frame. As the map grows, overlapping areas from different local tiles are blended, typically by simple averaging of overlapping pixels. This produces a coherent global seafloor mosaic that incorporates all loop closures and optimization corrections. The trajectory drawn over this map provides a clear visualization of the vehicles path and supports downstream tasks such as collision avoidance, mission planning, and situational awareness. Because the global map reflects the entire history of the mission, it can be used not only for real-time navigation but also for offline analysis and long term decision making.
\\ \\
This detailed architecture ties together all major concepts of the thesis. It provides a complete picture of how sensor data flows through the system, how each subsystem transforms that data, and how everything converges into the final optimized map and trajectory. The following chapters examine each of these components in depth, but this overview serves as the structural backbone for understanding the full SLAM pipeline.




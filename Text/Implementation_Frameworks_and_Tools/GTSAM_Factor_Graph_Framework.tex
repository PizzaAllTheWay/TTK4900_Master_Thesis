\subsection{GTSAM Factor Graph Framework}
The Georgia Tech Smoothing and Mapping (GTSAM) library is an open-source C++ framework for representing and solving estimation problems using factor graphs. Developed by Frank Dellaert and the BORG Lab at the Georgia Institute of Technology, GTSAM provides a flexible and efficient foundation for robotics and computer vision tasks involving probabilistic inference, including SLAM, structure from motion, and multi sensor fusion \cite{GTSAM_handbook}\cite{GTSAM_website}\cite{GTSAM_GitHub}.  
\\ \\
At its core, GTSAM models estimation problems as a product of small, local probability functions known as \textit{``factors''}, each depending on a subset of variables (eks, robot poses, velocities, landmarks, biases). These factors form a sparse graphical model, allowing efficient computation of the Maximum a Posteriori (MAP) estimate through variable elimination and nonlinear least squares optimization. This formulation provides a clear, modular representation of the SLAM problem while naturally capturing uncertainty and correlations between states.  
\\ \\
GTSAM performs optimization directly on manifolds using Lie group representations (\textit{SE(2)}, \textit{SE(3)}) for rotation and translation. This avoids the singularities of Euler angles or quaternions by updating states through tangent space retraction, ensuring consistent and stable 3D motion estimation and sensor fusion \cite{GTSAM_handbook}.
\\ \\
GTSAM features an advanced IMU preintegration module based on the work of Forster, Carlone, and Dellaert \cite{preintegration_camera_paper} and the earlier visual inertial framework by Lupton and Sukkarieh \cite{printegration_old}. It compresses high rate inertial data into a single relative motion factor, reducing computation while preserving accuracy. Using on manifold integration with bias correction and covariance propagation, it enables tight multi sensor fusion within the factor graph.
\\ \\
For real-time applications, GTSAM implements the iSAM2 algorithm \cite{iSAM2_paper}. iSAM2 maintains a Bayes tree data structure that enables local re-linearization and partial updates when new factors or variables are added. Instead of re-optimizing the entire graph, only the affected subtrees are updated, allowing smooth incremental estimation without large computational spikes. This makes iSAM2 ideal for online SLAM systems, where continuous sensor updates must be processed with bounded latency. The solver uses variable reordering (eks, constrained COLAMD) to minimize matrix fill in and preserve sparsity, ensuring scalability even for large scale mapping problems.  
\\ \\
In practice, GTSAM separates the model structure from the state representation. A \texttt{NonlinearFactorGraph} stores all factors, while a \texttt{Values} container holds the current estimates of unknown variables. New sensor measurements are added as typed factors, such as \texttt{BetweenFactor}, \texttt{PriorFactor}, or \texttt{ImuFactor}, each with its associated noise model. GTSAM then performs nonlinear optimization using solvers like Gauss-Newton or Levenberg-Marquardt. For online updates, the \texttt{iSAM2} interface provides continuous refinement of the solution, while preserving consistency between past and current estimates.  
\\ \\
In this work, GTSAM forms the core of the SLAM backend, handling factor graph construction, IMU preintegration, and real-time optimization through iSAM2. IMU, GNSS, and SLAM measurements are fused as factors that continuously refine the estimated trajectory and map. The modular design also allows straightforward extension with additional sensor types, such as visual or sonar and acoustic inputs. 
\\ \\
GTSAM is built on a strong academic foundation, combining key advances in probabilistic robotics and factor graph optimization \cite{Bayes_tree_for_SLAM_paper} \cite{iSAM_paper} \cite{iSAM2_paper}. Its modular design enables rapid prototyping of new factor types, testing of optimization methods, and integration with real world datasets. The framework has become a standard in research for its clarity, scalability, and rigorous mathematical grounding in SLAM and estimation.  
\\ \\
GTSAM is released under the BSD license, allowing modification and redistribution for academic and commercial use with proper attribution. The library and source code are publicly available at \url{https://gtsam.org/} \cite{GTSAM_website} and \url{https://github.com/borglab} \cite{GTSAM_GitHub}.  
\\ \\
In summary, GTSAM provides a robust foundation for real-time SLAM, combining factor graphs, manifold optimization, preintegration, and incremental solvers for efficient multi sensor fusion.


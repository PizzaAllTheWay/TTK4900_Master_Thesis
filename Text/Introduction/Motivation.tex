\subsection{Motivation}
Operating marine robots in coastal waters demands navigation systems that are resilient to environmental uncertainty and sensor degradation. Certain sensors such as GNSS and its performance degrades in coastal areas with tall structures, narrow passages, bridges, piers, or other large infrastructure, where signal blockage and multipath reflections reduce accuracy or cause intermittent loss. Dead reckoning drifts quickly, and static seafloor charts become outdated due to sediment movement, new constructions, and dynamic marine activity. For autonomous surface vessels such as microAmpere ASV, onboard mapping is essential for avoiding grounding, maintaining situational awareness, and enabling long duration autonomous operation without human supervision.
\\ \\
Side scan sonar offers a practical solution because its data reveals local seabed structure with high spatial consistency, but side scan sonar data is noisy, viewpoint dependent, and difficult to align without accurate motion estimates. Current real-time implementations of SSS SLAM show mixed results, while offline methods achieve good mapping accuracy but are not suited for real-time operation on autonomous vessels. The goals of this project are twofold. First, to study and outline a practical, lightweight SSS SLAM architecture, that can operate in real time on constrained embedded hardware while remaining robust on real datasets. Second, to implement this architecture on the microAmpere ASV, integrate it into the onboard perception stack, and evaluate its performance on real data and in real coastal environments. This shifts the work from theoretical and offline analysis toward a deployable, field ready SLAM capability for autonomous surface vessels.


